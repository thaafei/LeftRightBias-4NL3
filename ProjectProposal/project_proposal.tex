\documentclass[11pt]{article}

% Change "review" to "final" to generate the final (sometimes called camera-ready) version.
% Change to "preprint" to generate a non-anonymous version with page numbers.
\usepackage[]{acl}
\usepackage{times}
\usepackage{latexsym}


% For proper rendering and hyphenation of words containing Latin characters (including in bib files)
\usepackage[T1]{fontenc}

% This assumes your files are encoded as UTF8
\usepackage[utf8]{inputenc}
\usepackage{microtype}
\usepackage{inconsolata}
\usepackage{graphicx}
\usepackage{url}
\usepackage{booktabs}
\usepackage{amsmath, amssymb}


\title{Group 42 Progress Report:\\ Left and Right Leaning Bias Detection in Media Headlines}


\author{
  Ahren Chen, Fei Xie, Grace Xiao\\
  \texttt{\{chena125, xief17, xiaot13\}@mcmaster.ca}
}
\begin{document}
\maketitle

\section{Task Introduction}

Our project builds a Natural Language Processing model that classifies news/media headlines by political 
leaning (left vs. right). This task is significant because headlines heavily shape first impressions of 
an event, and an automatic “leaning” label can support media literacy tools, dataset exploration for 
researchers, and other media systems that want balanced coverage. It's challenging because political 
framing in headlines is often subtle and context-dependent: choices like which facts are revealed, 
what groups are named or hidden, and the use of terms can signal ideology without using obviously 
emotional language. Unlike sentiment analysis, our focus is ideological framing and viewpoint, not 
whether the headline sounds positive or negative. To keep our labels consistent, we'll write a 
short annotation guide with specific cues to look for, then have everyone in the team label the same starter 
set of headlines to line up our interpretations. Our goal is to reach agreement that's clearly better 
than chance, while recognizing that some headlines will still be genuinely hard to classify.

\section{Task Definition}

Task definition. Our dataset consists of media headlines, so each input example is a short piece of text. 
The task is a supervised classification problem: given a headline, the model predicts the headline's political 
leaning based on its framing. We use two classes (left leaning and right leaning) making this a binary 
classification setup. Each headline receives one label only (single-label), meaning it is annotated as 
either left or right (not both).

\section{Dataset}

\subsection{Plans for Collection}
We will use an existing annotated dataset \textit{allsides\_balanced\_news\_headlines-texts.txt} from \url{https://github.com/irgroup/Qbias/tree/main} . 
Each instance contains at minimum an English news \textbf{headline} and a political-leaning label (left/right), and may also include metadata such as the news outlet/source, event/topic title, topic tags, publication date, and a short article text/snippet (depending on the fields available in the release).

We will use only publicly available data for this course project and follow the dataset's license/usage terms.
We do not plan to perform additional web scraping. Prior to modeling, we will apply basic cleaning
(removing empty entries, de-duplication, and text normalization). We will also remove the center
class and formulate the task as a binary classification problem (left vs. right).

\subsection{Expected Dataset Size}
After removing the \texttt{center} class, we expect to use $N = 17{,}534$ labeled examples in total,
with $10{,}296$ left-leaning and $7{,}238$ right-leaning headlines. We will create train/dev/test splits with an approximate 80/10/10 ratio (with the exact proportions determined by the outlet-based grouping constraint).

As required, we provide three example data points with labels assigned by our group:
\begin{enumerate}
  \item \textbf{Event/Topic:} Gun Violence Over Fourth of July Weekend \\
        \textbf{Headline:} ``Chicago Gun Violence Spikes and Increasingly Finds the Youngest Victims'' \\
        \textbf{Outlet:} New York Times (News) \\
        \textbf{Group label:} \texttt{left}
  \item \textbf{Event/Topic:} Gun Violence Over Fourth of July Weekend \\
        \textbf{Headline:} ``Dozens of shootings across US mark bloody July 4th weekend'' \\
        \textbf{Outlet:} New York Post (News) \\
        \textbf{Group label:} \texttt{right}
  \item \textbf{Event/Topic:} Yellen Warns Congress of ``Economic Recession'' if Debt Ceiling Isn't Raised \\
        \textbf{Headline:} ``Federal Government Will Run Out of Cash on Oct.\ 18 If Debt Ceiling Isn't Raised: Treasury Secretary'' \\
        \textbf{Outlet:} The Epoch Times \\
        \textbf{Group label:} \texttt{right}
\end{enumerate}


\section{Team Contract}
The team contract outlines the expectations of the team for the duration of the project. Submission of this project proposal implies that all members listed in the author section has read and agreed to the following contract.
\subsection{Team Purpose}
Our purpose is to apply the content and knowledge we learned during the SFWRENG 4NL3:
Natural Language Processing class towards our project, classifying if a given news headline can indicate if the article is more right or left leaning.
\subsection{Team Goals}
  \begin{itemize}
    \item Complete all deliverables on time and to a high degree of quality.
    \item Maintain equal participation and contribution from all members
    \item Develop skills in collaboration, problem-solving, and accountability.
    \item Create an interesting project based on the course content we learn in class.
  \end{itemize}
\subsection{Communication Norms}
Primary platform: Discord \\
Response expectation: Reply to discord messages within 24 hours.\\
Decision-Making: Strive for consensus when possible; else, majority vote will be used.\\
\subsection{Ground Rules and Expectations}
  \begin{itemize}
    \item Be respectful of all ideas and opinions.
    \item Come prepared for meetings.
    \item Complete/submit assigned work on time.
    \item It is up to the individual assigned to the task to inform the rest of the team if the task cannot be completed on time.
    \item All team members are expected to attend tutorial time [Wednesday, 4:30-5:20 PM], and to inform the team beforehand if they cannot make it.
  \end{itemize}
\subsection{Conflict Resolution}
We recognize that conflicts may arise, particularly around workload distribution. To handle
these situations:
\begin{enumerate}
  \item Direct Conversation: The concerned member(s) will address the issue privately
and respectfully with the person involved.
  \item Group Discussion: If the problem continues, the whole team will discuss it openly
and constructively, focusing on solutions (e.g., redistributing tasks, adjusting
timelines).
  \item Clear Expectations: If a team member repeatedly does not contribute,
responsibilities will be formally reassigned, and the lack of contribution will be
documented.
\item Escalation: If the issue remains unresolved, the team will notify the course
instructor/TA with a written summary of the situation.
\end{enumerate}
\end{document}
